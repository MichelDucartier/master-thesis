\section{Introduction}
\label{sec:introduction}

\subsection{Problem overview and research gap}

Begin with a clear statement of the broader context and significance of your research area. Present recent statistics and evidence demonstrating the problem's impact. Follow this with current approaches to solving the problem, their limitations, and the specific gaps in existing solutions that your work addresses. Build a clear narrative that leads to why your research is necessary.

\subsection{Literature review}

Structure your literature review chronologically or thematically, depending on your field. Start with foundational work that established key concepts. Progress to recent developments and state-of-the-art approaches. Critically analyze existing methods, comparing their strengths and limitations. Focus particularly on works that directly relate to your research question. 

\subsection{Research objectives}\label{sec:aim}

State your primary research aim in one clear, focused sentence. Follow this with specific objectives that break down how you will achieve this aim. Each objective should be:
\begin{enumerate}
    \item To systematically investigate [specific aspect] (addressed in Section \ref{sec:methods})
    \item To develop and implement [specific method] (addressed in Section \ref{sec:results1})
    \item To evaluate and validate [specific outcome] (addressed in Section \ref{sec:results2})
    \item To compare [specific comparison] (addressed in Section \ref{sec:results3})
\end{enumerate}

\subsection{Contributions}

Conclude your introduction by explicitly stating your novel contributions. Explain how your work advances the field beyond existing approaches. Describe the practical implications of your research and its potential impact.
