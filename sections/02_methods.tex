\section{Methods}
\label{sec:methods}

\subsection{Study design}

Describe your overall approach and provide the rationale for your chosen methodology. Your description should be detailed enough that other researchers could reproduce your work.

\subsection{Data}

\subsubsection{Dataset characteristics}

Describe your data sources, collection methods, and time period. Define all inclusion and exclusion criteria. Specify measures taken to ensure data quality. Present this information in a clear table format:

\begin{table}[htbp]
    \centering
    \caption{Present your dataset characteristics here. Include total sample size, training/validation/test split, and relevant demographic information.}
    \label{tab:dataset}
    \begin{threeparttable}
    \resizebox{0.4\textwidth}{!}{%
    \begin{tabular}{llr}
    \toprule
    \textbf{Characteristic} & \textbf{Category} & \textbf{Count (\%)}\tnote{1} \\
    \midrule
    Sample Size & Total & 1000 (100.0\%)\\
               & Training & 700 (70.0\%) \\
               & Validation & 150 (15.0\%) \\
               & Test & 150 (15.0\%) \\
    \midrule
    Gender & Male & 520 (52.0\%) \\
           & Female & 460 (46.0\%) \\
           & Other & 20 (2.0\%) \\
    \midrule
    Age Group\tnote{2} & 18-24 & 200 (20.0\%) \\
             & 25-34 & 350 (35.0\%) \\
             & 35-44 & 250 (25.0\%) \\
             & 45-54 & 150 (15.0\%) \\
             & 55+ & 50 (5.0\%) \\
    \midrule
    Education\tnote{3} & High School & 300 (30.0\%) \\
             & Bachelor's & 450 (45.0\%) \\
             & Master's & 200 (20.0\%) \\
             & Ph.D. & 50 (5.0\%) \\
    \bottomrule
    \end{tabular}%
    }
    \begin{tablenotes}
      \small
      \item[1] All percentages are rounded to one decimal place
      \item[2] Age groups are presented in years
      \item[3] Education represents the highest level completed
    \end{tablenotes}
    \end{threeparttable}
\end{table}

\subsection{Methodology}

\subsubsection{Preprocessing pipeline}

Document each step of your preprocessing pipeline in sequence. Explain the rationale behind each decision made during preprocessing. Include any parameters or thresholds used.

\subsubsection{Model architecture}

Present your model's complete architecture. Include a detailed diagram showing all components and their connections. Your diagram should look similar to this:

\begin{figure}[htbp]
    \centering
    \includegraphics[width=0.8\textwidth]{example-image}
    \caption{Create a detailed diagram of your model architecture. Label all components,
             connections, and data flows. Include input and output dimensions.}
    \label{fig:modelarch}
\end{figure}

\subsection{Implementation details}

\subsubsection{Software and hardware}

List all software tools, libraries, and versions used. Specify the computing resources required to reproduce your work. Include any special hardware requirements.

\subsubsection{Hyperparameters}

Document all hyperparameters used in your model. Explain how you selected these values. Describe any optimization or tuning processes used.

\subsection{Evaluation methodology}

\subsubsection{Metrics}

Define each evaluation metric used in your study. Explain why these metrics are appropriate for your research questions. Include formulas where necessary.

\subsubsection{Statistical analysis}

Describe all statistical tests performed. Include power calculations where applicable. Specify significance levels and confidence intervals used.
