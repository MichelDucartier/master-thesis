\begin{abstract}

Clinical decision-making is inherently multimodal, relying on diverse data sources such as medical imaging and electronic health records. Current open-source frameworks for multimodal AI research often lack the necessary flexibility and extensibility to accommodate the unique requirements of clinical applications.

Furthermore, reproducibility is a cornerstone of scientific research, ensuring that models can be trusted and verified. In the field of AI, the "Reproducibility Crisis" is exacerbated by the lack of open-source code, datasets and replicable experimental settings. These shortcomings hinder both the reproducibility of research and the ability to extend upon it.

To address these challenges, we introduce MultiMeditron, a robust pipeline built on the Llava architecture, which emphasizes extensibility across multiple modalities, clean code, thorough documentation, and reproducible settings.

Our goal is twofold. First, we aim at building a reproducible framework for AI research that integrates modular, scalable pipelines and transparent practices. Second, we present a first suite of multimodal medical models that highlights the capabilities of the MultiMeditron framework.

\end{abstract}

% keywords can be removed
% \keywords{First keyword \and Second keyword \and Additional keywords (max 5--6 in total). Order them from most to least specific, separated by `\mintinline{latex}{\and}'. These keywords should help readers find your work in database searches.}


\clearpage
\graphicalabstract[
    % landscape,  % uncomment if you need landscape
    width=0.9\textwidth,
    caption={Diagram of the MultiMeditron framework}  % Fill the caption here
]{images/MultiMeditron.png}  % Fill the filename here

