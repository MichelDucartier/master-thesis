\begin{abstract}

% Your abstract should be under 350 words and (optionally) structured into distinct sections. Do not include citations or abbreviations unless absolutely necessary. Each section should accomplish the following:
%
% \textbf{BACKGROUND.} Begin with a statement of the overall problem that affects a significant population. Follow this with a brief description of existing solutions or approaches to this problem. Then identify the specific limitations or gaps in these current solutions. Conclude with a brief introduction to your proposed solution that addresses these limitations.
%
% \textbf{AIM.} Write one clear, concise sentence that states your primary research goal. If your study has specific sub-objectives, list no more than three of the most important ones here.
%
% \textbf{METHODS.} Describe your study design and setting. State when and where you collected your data. Specify your sample size and describe the key characteristics of your study population. Outline your primary methods or interventions. Define your main outcome measures. Present this information in a logical sequence that a reader can follow.
%
% \textbf{RESULTS.} Present your key findings with specific numerical results. Include measures of statistical significance, such as p-values and confidence intervals, where appropriate. Compare your results with existing methods or baselines. For example: \textit{Our model achieves 91.2\% (CI\textsubscript{95}: 89.5--93.2) sensitivity and 80\% (CI\textsubscript{95}: 74.1--86.5) specificity, representing a 5\% and 7\% improvement over current methods respectively (p<0.05).}
%
% \textbf{CONCLUSION.} Summarize the main implications of your findings and their potential impact on the field. If appropriate, include a brief statement about future directions or applications of your work.
%

Reproducibility in scientific research is crucial to build trust and verifiable claims.

\end{abstract}

% keywords can be removed
\keywords{First keyword \and Second keyword \and Additional keywords (max 5--6 in total). Order them from most to least specific, separated by `\mintinline{latex}{\and}'. These keywords should help readers find your work in database searches.}


\clearpage
\graphicalabstract[
    % landscape,  % uncomment if you need landscape
    width=0.9\textwidth,
    caption={This is a description of the graphical abstract (optional)}  % Fill the caption here
]{example-image-a4-numbered.pdf}  % Fill the filename here

